% Tout ce qui est mis derri�re un � % � n'est pas vu par LaTeX
% On appelle cela des � commentaires �.  Les commentaires permettent de
% commenter son document - comme ce que je suis en train de faire
% actuellement - et de cacher du code - cf. la ligne \pagestyle.

%\documentclass[a4paper, titlepage]{book}
\documentclass[a4paper, 12pt, twoside]{book}
% Options possibles : 10pt, 11pt, 12pt (taille de la fonte)
%                     oneside, twoside (recto simple, recto-verso)
%                     draft, final (stade de d�veloppement)
%                     tutelage, noticeable (\maketitle prend une page
%                                             � part ou non)
\usepackage[latin1]{inputenc} % LaTeX, comprends les accents !
\usepackage[T1]{fontenc}      % Police contenant les caract�res fran�ais
\usepackage{geometry}         % D�finir les marges
\usepackage[frenchb]{babel}
%\usepackage[francais]{babel}  % Placez ici une liste de langues, la
                              % derni�re �tant la langue principale
%\usepackage[babel]{csquotes}  % Ces lignes permettent d'utiliser simplement les guillemets sans
%\MakeAutoQuote{�}{�}          % espaces, le package csquotes s'occupera de l'espacement.
%\usepackage{makeidx}          % permet de cr�er un index
\usepackage{textcomp}
%\usepackage[french, refpages]{gloss} % pour g�n�rer un glossaire
% Hyperlien pour la table des mati�re (� compiler 2 fois)
\usepackage[
%   ps2pdf,
   pdftex,
   colorlinks,
   linkcolor=black,
   ]{hyperref}
% \pagestyle{headings}        % Pour mettre des ent�tes avec les titres
                              % des sections en haut de page

                              % Les param�tres du titre : titre, auteur, date
\title{
Flat Assembler 1.66\\
\begin{normalsize}Manuel du Programmeur\end{normalsize}
\\
}

\author{
Thomas Grysztar\\
\\
\\
\\
\\
\\
\\
\\
\\
\\
\\
\begin{small}version Fran�aise\end{small}
\\
\begin{small}Nicolas Djurovic\end{small}
}

\date\today
%\date{}                      % La date n'est pas requise (la date du
                              % jour de compilation est utilis�e en son
                              % absence

%Il faut inclure dans le pr�ambule du document (avant le \begin{document}) :
%\makeindex
%\makeglossary

\begin{document}

\maketitle                    % Faire un titre utilisant les donn�es
                              % pass�es � \title, \author et \date

%\frontmatter                  % Prologue

%\chapter{Contenus}

\tableofcontents
%\mainmatter                   % On passe aux choses s�rieuses

\index{Introduction}
\chapter{Introduction}
Ce chapitre contient toutes les informations importantes dont vous
aurez besoin pour commencer � utiliser flat assembler. Si vous
�tes un programmeur exp�riment� en assembleur, vous devriez lire
au moins ce chapitre avant d'utiliser ce compilateur.

\index{Introduction!Survol du Compilateur}
\section{Survol du Compilateur}
Flat assembler est un compilateur rapide g�n�rant du langage
machine pour les processeurs x86, faisant une multitude de
passages pour optimiser la taille du code g�n�r�. Il est
auto-compilable et des versions pour diff�rentes plates-formes
sont fournies. Ces versions ont �t� con�ues pour �tre utilis� �
partir de la ligne de commande et leur fonctionnement sur les
diff�rentes plates-formes devrait �tre identique.

Ce document d�crit aussi la version IDE con�ue pour le syst�me
Windows qui utilise une interface graphique, et qui a un �diteur
int�gr�, au lieu de la version console. Mais d'un point de vue
compilation, il a exactement les m�mes fonctionnalit�s que la
version console, et les parties suivantes (� partir du chapitre
1.2 de ce document) seront identiques � toutes les versions de
flat assembler. L'ex�cutable de la version IDE est
\texttt{fasmw.exe}, alors que la version ligne de commande est
\texttt{fasm.exe}

\index{Introduction!Pr�-requis syst�me}
\subsection{Pr�-requis syst�me}
Toutes les versions utilisent un processeur x86/32 bits (un 80386
minimum), cependant ils peuvent aussi concevoir des programmes
pour les architectures x86/16 bits. La version console de Windows
utilise n'importe quel syst�me d'exploitation Windows 32 bits,
alors que la version avec interface graphique requiert une version
4.0 ou plus du syst�me, il fonctionnera donc sur tous syst�mes
compatibles avec Windows 95 et sup�rieure.

Les exemples fournies avec cette version implique que la variable
d'environnement \texttt{INCLUDE} soit d�finie par le chemin du
r�pertoire \texttt{include} de flat assembler. Si cette variable
existe d�j� dans votre syst�me et contient des chemins utilis�s
par d'autres programmes, vous n'avez qu'� lui ajouter le nouveau
chemin (les diff�rents chemins doivent �tre s�par�s par des
points-virgule). Si vous ne voulez pas d�finir ce type de
variable, ou si vous ne savez pas comment, vous pouvez le d�finir
pour la version avec interface graphique de flat assembler en
�ditant le fichier \texttt{fasmw.ini} qui est dans le m�me
r�pertoire, ce fichier est cr�� automatiquement d�s le 1$^{er}$
lancement de \texttt{fasmw.exe}, mais vous pouvez aussi le cr�er
par vous-m�me, dans ce cas vous devrez ajouter la valeur
\texttt{Include} dans la section \texttt{Environment}. Par
exemple, si vous avez install� les fichiers flat assembler dans le
r�pertoire \texttt{c$:$$\backslash$fasmw}, vous devrez ins�rer ces
deux lignes suivantes dans votre fichier
\texttt{c$:$$\backslash$fasmw$\backslash$fasmw.ini} :
\begin{verbatim}[Environment]
Include = c:\fasmw\include
\end{verbatim}
Si vous ne d�finissez pas la variable d'environnement
\texttt{INCLUDE} correctement, vous devrez mettre manuellement le
chemin complet des includes Windows dans chaque programme que vous
allez compiler.

\index{Introduction!Utilisation du compilateur}
\subsection{Utilisation du compilateur}
Pour commencer � utiliser flat assembler, double-cliquer sur
l'ic�ne du fichier \texttt{fasmw.exe}, ou d�placez l'ic�ne de
votre fichier source dessus (drag \& drop). Plus tard vous pourrez
aussi ouvrir des fichiers sources avec la commande \emph{Open} du
menu \emph{File}, ou en d�pla�ant les fichiers dans l'�diteur.
Vous pouvez avoir plusieurs fichiers ouverts simultan�ment, chacun
�tant repr�sent� par un onglet en haut de l'�diteur. Pour
s�lectionner le fichier � �diter vous n'avez qu'� cliquer sur
l'onglet correspondant avec le clique gauche de votre souris. Par
d�faut le compilateur g�re le fichier que vous �tes en train
d'�diter, mais vous pouvez forcer le compilateur � g�rer un
fichier en particulier en cliquant sur l'onglet appropri� avec le
bouton droit de votre souris et en s�lectionnant \emph{Assign}. Un
seul fichier peut �tre assign� � la fois.

Quand votre fichier est pr�t � �tre compil�, s�lectionnez la
commande \emph{Compile} dans le menu \emph{Run}. Un r�sum� de la
compilation sera affich� si celle-ci est r�ussie, sinon l'erreur
sera affich�. Dans ce r�sum� se trouvera une information sur le
nombre de passages effectu�, combien de temps cela a pris, et
combien d'octets ont �t� �crits dans le fichier final. Dans ce
r�sum� vous avez aussi un emplacement appel� \emph{Display} dans
lequel vous trouverez tous les messages affich�s gr�ce � la
directive \texttt{display} (voir section 2.2.5). Si l'erreur est
li�e � une ligne sp�cifique de votre code source, elle s'affichera
alors dans le champ texte \emph{Instruction} contenant
l'instruction qui a caus� l'erreur et uniquement si une erreur
appara�t lors de cette phase du pr�-processeur (autrement ce champ
reste vide). Ce r�sum� contient aussi le champ \emph{Source}, qui
est une liste affichant les lignes de toutes les sources qui se
r�f�rent � cette erreur, lorsque l'on clique sur l'une des lignes
de cette liste, la ligne ou se trouve l'erreur sera
automatiquement s�lectionn�e dans l'�diteur (si le fichier n'est
pas ouvert dans l'�diteur, il sera ouvert automatiquement).

La commande \emph{Run} lance aussi le compilateur et, en cas de
r�ussite de la compilation, il lancera la programme compil� si
celui-ci est un format qui peut-�tre lanc� dans un environnement
Windows, autrement vous aurez un message vous indiquant que ce
fichier ne peux pas �tre lanc�. Si une erreur appara�t, le
compilateur affichera une information identique � celle que l'on
trouve avec la commande \emph{Compile} que l'on a utilis�.

Si le compilateur vous indique qu'il n'y a pas assez de m�moire,
vous pouvez accro�tre la m�moire normalement allou� via la bo�te
de dialogue \emph{Compiler setup}, dans le menu \emph{Options}.
Dans ce menu vous pouvez donner le nombre de kilo-octets que le
compilateur devra utiliser ainsi que la priorit� du compilateur
dans les processus syst�me.

\index{Introduction!Ex�cuter le compilateur via la ligne de
commande}
\subsection{Ex�cuter le compilateur via la ligne de commande}
Pour compiler � partir de la ligne de commande vous devez ex�cuter
l'executable \texttt{fasm.exe} avec deux param�tres - le nom de
votre ficher source et le nom du fichier destination. Si aucun
second param�tre n'est donn�, le nom sera g�n�r� automatiquement
(en fonction du nom du fichier source). Apr�s l'affichage de
quelques courtes informations � propos du nom du programme et de
sa version, le compilateur lira les donn�es du fichier source et
lancera la compilation. Quand il aura fini, le compilateur �crira
le code g�n�r� dans le fichier final et affichera un r�sum� du
processus de compilation; sinon il affichera les erreurs qui sont
apparues.

Le fichier source devra �tre un fichier texte et peut �tre cr��
avec n'importe quel �diteur texte. Les caract�res de fin de ligne
sont aussi bien accept�s par les standards DOS et Unix, les
tabulations sont consid�r�es comme des caract�res "espace".

Vous pouvez aussi inclure dans la ligne de commande le param�tre
\texttt{-m} suivi par un num�ro, cela permettra � flat assembler
de conna�tre le nombre de kilo-octets maximal qu'il doit utiliser.
Pour la g�n�ration du fichier DOS seulement, cette option limite
le nombre d'utilisation de la m�moire �tendue. L'option
\texttt{-p} suivi par un num�ro peut �tre utilis� pour sp�cifier
le nombre maximum de passages de l'assembleur. Si le code ne peut
�tre g�n�r� dans le montant de passages sp�cifi�s, l'assemblage se
terminera avec un message d'erreur. La valeur maximum de cette
option est 65536, alors que par d�faut, si cette option n'est pas
dans la ligne de commande, elle est de 100.

Il n'y a aucune option qui pourrait affecter la sortie du
compilateur, flat assembler demande seulement le code source comme
information vraiment utile. Par exemple, pour sp�cifier un format
de sortie diff�rent, vous l'indiquez en utilisant la directive
\texttt{format} au d�but de votre code source.

\index{Introduction!Messages du compilateur en ligne de commande}
\subsection{Messages du compilateur en ligne de commande}
Apr�s le succ�s de la compilation le compilateur affiche donc un
r�sum�. Il nous donne le nombre de passages effectu�, combien de
temps cela a pris, et combien d'octets ont �t� �crits dans le
fichier final. Voici un exemple d'un r�sum� de la compilation :
\begin{verbatim}flat assembler  version 1.66
38 passes, 5.3 seconds, 77824 bytes.
\end{verbatim}
En cas d'erreur pendant la phase de compilation, le programme
affiche un message d'erreur. Par exemple, quand le compilateur ne
trouve pas le fichier source, il affichera le message suivant :
\begin{verbatim}flat assembler  version 1.66
error: source file not found.
\end{verbatim}
Si l'erreur se trouve � un endroit particulier dans le code
source, la ligne qui a caus� l'erreur sera alors affich�e. Le
num�ro de la ligne vous est donn� pour que vous puissiez trouver
l'erreur, par exemple :
\begin{verbatim}flat assembler  version 1.66
example.asm [3]:
        mob     ax,1
error: illegal instruction.
\end{verbatim}
A la 3$^{eme}$ ligne du fichier \texttt{example.asm}, le
compilateur a rencontr� une instruction inconnue. Lorsque la ligne
qui a caus� une erreur contient une macro-instruction, alors la
ligne d�finissant cette macro et ayant g�n�r�e une instruction
incorrect est affich� :
\begin{verbatim}flat assembler  version 1.66
example.asm [6]:
        stoschar 7
example.asm [3] stoschar [1]:
        mob     al,char
error: illegal instruction.
\end{verbatim}
A la 6$^{eme}$ ligne du fichier \texttt{example.asm}, la
macro-instruction a g�n�r� une instruction inconnue dans la
1$^{ere}$ ligne de sa d�finition.

\index{Introduction!Formats de sortie}
\subsection{Formats de sortie}
Par d�faut, quand aucune directive \texttt{format} n'est donn�
dans le code source, flat assembler g�n�re des codes
d'instructions sur la sortie, cr�ant ainsi un fichier binaire. Il
g�n�re par d�faut du code 16 bits, mais vous pouvez le mettre en
mode 16 bit ou en mode 32 bits simplement en utilisant les
directives \texttt{use16} ou \texttt{use32}. Quelques formats de
sortie, lorsqu'ils sont s�lectionn�s, se mettent en mode 32 bit -
vous trouverez plus d'informations sur les formats utilisable dans
la section 2.4.

En fonction du format de sortie s�lectionn�, le compilateur choisi
automatiquement le type d'extension du fichier de destination.

Chaque code de sortie est toujours dans l'ordre dans lequel il a
�t� entr� dans le fichier source.

\index{Introduction!Syntaxe de l'assembleur}
\section{Syntaxe de l'assembleur}
L'information fournie ci-dessous est pr�vue principalement pour
les programmeurs qui ont utilis�s d'autres compilateurs
auparavant. Si vous �tes d�butant, vous devriez jeter un oeil sur
les tutoriaux de la programmation en assembleur.

Flat assembler utilise par d�faut la syntaxe Intel pour les
instructions d'assemblage, cependant vous pouvez l'adapter en
utilisant les possibilit�s du pr�-processeur (macro-instructions
et constantes symboliques). Aussi il a son propre ensemble de
directives - les instructions pour le compilateur.

Tous les symboles d�finis dans les sources respectent la
diff�rence entre majuscule/minuscule.
\begin{table}[!h]
    \begin{center}
        \begin{tabular}{|c|c|c|}
        \hline
        Op�rateur&Bits&Octets\\
        \hline\hline
        \verb"byte"&8&1\\
        \hline
        \verb"word"&16&2\\
        \hline
        \verb"dword"&32&4\\
        \hline
        \verb"fword"&48&6\\
        \hline
        \verb"pword"&48&6\\
        \hline
        \verb"qword"&64&8\\
        \hline
        \verb"tbyte"&80&10\\
        \hline
        \verb"tword"&80&10\\
        \hline
        \verb"dqword"&128&16\\
        \hline
        \end{tabular}
    \end{center}
    \caption{Taille des op�rateurs}
\end{table}
%\begin{center}
%    Table 1.1: Taille des op�rateurs.
%\end{center}

\begin{table}[!h]
    \begin{center}
        \begin{tabular}{|c|c||c c c c c c c c|}
        \hline
        Types&Bits&&&&&&&&\\
        \hline\hline
        &8&\verb"al"&\verb"cl"&\verb"dl"&\verb"bl"&\verb"ah"&\verb"ch"&\verb"dh"&\verb"bh"\\
        General&16&\verb"ax"&\verb"cx"&\verb"dx"&\verb"bx"&\verb"sp"&\verb"bp"&\verb"si"&\verb"di"\\
        &32&\verb"eax"&\verb"ecx"&\verb"edx"&\verb"ebx"&\verb"esp"&\verb"ebp"&\verb"esi"&\verb"edi"\\
        \hline
        Segment&16&\verb"es"&\verb"cs"&\verb"ss"&\verb"ds"&\verb"fs"&\verb"gs"&&\\
        \hline
        Control&32&\verb"cr0"&&\verb"cr2"&\verb"cr3"&\verb"cr4"&&&\\
        \hline
        Debug&32&\verb"dr0"&\verb"dr1"&\verb"dr2"&\verb"dr3"&&&\verb"dr6"&\verb"dr7"\\
        \hline
        FPU&80&\verb"st0"&\verb"st1"&\verb"st2"&\verb"st3"&\verb"st4"&\verb"st5"&\verb"st6"&\verb"st7"\\
        \hline
        MMX&64&\verb"mm0"&\verb"mm1"&\verb"mm2"&\verb"mm3"&\verb"mm4"&\verb"mm5"&\verb"mm6"&\verb"mm7"\\
        \hline
        SSE&128&\verb"xmm0"&\verb"xmm1"&\verb"xmm2"&\verb"xmm3"&\verb"xmm4"&\verb"xmm5"&\verb"xmm6"&\verb"xmm7"\\
        \hline
        \end{tabular}
    \end{center}
    \caption{Les registres}
\end{table}
%\begin{center}
%    Table 1.2: Les registres.
%\end{center}

\index{Introduction!Syntaxe des Instructions}
\subsection{Syntaxe des Instructions}
En assembleur, les instructions sont s�par�es par des sauts de
ligne et il n'y a qu'une instruction par ligne. Si cette ligne
contient un point-virgule, sauf si elle se trouve dans un texte
d�limit� par des guillemets, le reste de la ligne est consid�r�
comme un commentaire et ne sera donc pas pris en compte par le
compilateur. Si une ligne se termine avec le caract�re
$\backslash$ (un commentaire peut suivre ce caract�re), la ligne
suivante se rattache � celle-ci.

Chaque ligne dans le source est une suite d'�l�ments qui peut �tre
un des trois types connus. Il y a les caract�res symboliques, ce
sont des caract�res sp�ciaux qui sont uniques m�me lorsqu'ils ne
sont pas espac�s des autres. Chaque
\texttt{+-*/=<>\(\)[]}\verb"{"\verb"}"\verb":"\texttt{,|\&}\verb"~"\texttt{\#`}
est un caract�re symbolique. Pour les autres caract�res, s�par�s
des autres �l�ments par des espaces ou par des caract�res
symboliques est un caract�re symbolique. Si le 1$^{er}$ caract�re
est un �'� ou un �"�, tous les caract�res suivants, m�me les
caract�res sp�ciaux, deviennent une cha�ne qui devra se terminer
par le m�me caract�re utilis� au d�but (apostrophe ou guillemet) -
cependant si on trouve deux de ces caract�res identiques qui se
suivent (sans aucun caract�re entre eux), ils sont int�gr�s dans
la cha�ne de caract�res sans arr�ter la d�limitation de celle-ci.
Les symboles autres que les caract�res symboliques ou que les
cha�nes de caract�res peuvent �tre utilis�s comme �tant des noms,
aussi d�finies comme noms symboliques.

Chaque instruction est repr�sent�e par un mn�monique (nom
symbolique) et un nombre vari� d'op�randes (constante, symbolique
ou pas, ou une variable) s�par� par des virgules. L'op�rande peut
�tre un registre, une valeur imm�diate ou une donn�e adress�e en
m�moire, il peut aussi �tre pr�c�d� par un op�rateur de taille
pour d�finir ou augmenter sa taille (Tableau 1.1). Les noms des
registres disponibles sont dans le Tableau 1.2, leur taille ne
peut �tre augment�e. Les valeurs imm�diates peuvent �tre indiqu�es
par une expression num�rique.

Quand l'op�rande est une donn�e en m�moire, l'adresse de cette
donn�e (ou expression num�rique mais dans un registre) doit �tre
mise entre crochets (\texttt{[ ]}) ou est pr�c�d� par l'op�rateur
\texttt{ptr}. Par exemple l'instruction \texttt{mov eax,3} met la
valeur imm�diate 3 dans le registre \texttt{eax}, l'instruction
\texttt{mov eax,[7]} met la valeur (32 bit) se trouvant �
l'adresse 7 dans le registre \texttt{eax} et l'instruction
\texttt{mov byte [7],3} met la valeur imm�diate 3 dans l'octet se
trouvant � l'adresse 7, on peut aussi l'�crire \texttt{mov byte
ptr 7,3}. Pour indiquer quel registre de segment doit �tre utilis�
pour l'adressage, son nom suivi par deux points (\verb":") devra
�tre mis juste avant la valeur de l'adresse (entre les crochets ou
apr�s l'op�rateur \texttt{ptr}) :
\begin{verbatim}    mov word [es:bx],ax
\end{verbatim}

\index{Introduction!D�finitions des donn�es}
\subsection{D�finitions des donn�es}
Pour d�finir une donn�e ou r�server de la place pour elle, il faut
utiliser une des directives donn�e dans le Tableau 1.3. La
directive de d�finition de donn�e doit �tre suivie par une ou
plusieurs expressions num�riques s�par�es par des virgules.
Suivant la directive utilis�e ces expressions d�finissent la
taille des valeurs pour les donn�es. Par exemple \texttt{db 1,2,3}
d�fini trois octets de valeur 1, 2 et 3.

Les directives \texttt{db} et \texttt{du} acceptent aussi les
cha�nes de caract�res de n'importe quelle taille, qui seront
converties en octet (8 bits) quand \texttt{db} sera utilis� et en
word (16 bits) avec la valeur haute �gale � z�ro (0) quand
\texttt{du} est utilis�. Par exemple \texttt{db 'abc'} d�finira
les 3 octets de valeurs 61, 62 et 63.

Les directives \texttt{dp} et \texttt{df} acceptent les valeurs
d�finies comme �tant deux expressions num�riques s�par�s par
(\verb":"), la premi�re valeur deviendra un mot (\texttt{word}) de
poids fort, tandis que la seconde valeur sera un double word de
poids faible de la valeur du pointeur en-dehors du segment
courant. La directive \texttt{dd} accepte aussi ce type de
pointeur compos� de deux valeurs word s�par�es par (\verb":") et
\texttt{dt} accepte comme valeur un word et un quadruple word
s�par�es par (\verb":"), le quad word est stock� en premier. La
directive \texttt{dt} avec une seule expression comme param�tre
accepte les valeurs en virgule flottante et cr�� les donn�es en
FPU (Floating Point Unit/Unit� de calcul en virgule flottante)
format �tendue double pr�cision.

Chacune de ces directives permet l'utilisation de l'op�rateur
sp�cial \texttt{dup} permet de faire des copies multiples de
valeur. Le multiplicateur doit pr�c�d� cette directive et les
valeurs � dupliquer sont apr�s celle-ci - on peut mettre plusieurs
valeurs, tant qu'elles sont s�par�es par des virgules et
d�limit�es par des parenth�ses, comme \texttt{db 5 dup (1,2)}, on
d�fini 5 copies des deux octets soit, apr�s la phase de
pr�-processus, \texttt{db 1,2,1,2,1,2,1,2,1,2}.

\texttt{file} est une directive sp�ciale et sa syntaxe est
diff�rente. Cette directive inclut les octets d'un fichier et doit
�tre suivi par le nom du fichier entre apostrophes, en option on
peut ajouter une expression num�rique indiquant l'offset du
fichier pr�c�d� par (\verb":"), et on peut encore ajouter une
derni�re option permettant d'indiquer le nombre d'octets �
inclure, mettre une virgule avant la valeur (s'il n'y a pas de
valeur, toutes les donn�es lues � partir de l'offset jusqu'� la
fin du fichier seront inclus). Par exemple \texttt{file
'data.bin'} ins�rera toutes les donn�es du fichier binaire et
\texttt{file 'data.bin'}\verb":"\texttt{10h,4} ins�rera seulement
4 octets � partir de l'offset/adresse 10h.
\begin{table}[!h]
    \begin{center}
        \begin{tabular}{|c||c|c|}
        \hline
        Taille&D�finir les&R�server les\\
        (octets)&donn�es&donn�es\\
        \hline\hline
        1&\verb"db"&\verb"rb"\\
        &\verb"file"&\\
        \hline
        2&\verb"dw"&\verb"rw"\\
        &\verb"du"&\\
        \hline
        4&\verb"dd"&\verb"rd"\\
        \hline
        6&\verb"dp"&\verb"rp"\\
        &\verb"df"&\verb"rf"\\
        \hline
        8&\verb"dq"&\verb"rq"\\
        \hline
        10&\verb"dt"&\verb"rt"\\
        \hline
        \end{tabular}
    \end{center}
    \caption{Directives des donn�es}
\end{table}

La directive de r�servation de donn�e doit �tre suivi par une
seule expression num�rique, et cette valeur d�finie combien de
zones m�moires, d'une taille sp�cifi�e, doit �tre r�serv�e. Toutes
les directives de d�finition des donn�es acceptent aussi la valeur
\texttt{?} qui permet d'initialiser une zone m�moire avec
n'importe quelle valeur, comme \texttt{label db ?} et l'effet est
identique � la directive de r�servation de donn�es. La donn�e non
initialis�e ne sera pas incluse dans le fichier final, donc ces
valeurs devront toujours �tre consid�r�es comme inconnues.

\index{Introduction!Constantes et labels}
\subsection{Constantes et labels}
Dans les expressions num�riques vous pouvez aussi utiliser des
constantes ou des labels � la place des nombres. Pour d�finir une
constante ou un label vous devrez utiliser des directives
sp�cifiques. Chaque label doit �tre d�fini une seule fois et sera
accessible de n'importe quel endroit du source (m�me avant qu'il
soit d�fini). Une constante peut �tre red�finie plusieurs fois,
mais sera accessible uniquement apr�s sa d�finition et sera
toujours �gale � la derni�re valeur red�finie avant l'endroit ou
elle est utilis�e. Quand une constante est d�finie une seule fois
dans le source il est, comme pour le label, accessible de
n'importe ou.

La d�finition d'une constante se compose du nom de la constante
suivi par le caract�re "\texttt{=}" et d'une expression num�rique,
qui apr�s calcul deviendra la valeur de cette constante. Cette
valeur est toujours calcul�e au moment de la d�finition de la
constante. Par exemple vous pouvez d�finir la constante
\texttt{count} en utilisant la directive \texttt{count = 17}, et
ainsi l'utiliser avec les instructions de l'assembleur comme
\texttt{mov cv,count} - qui deviendra \texttt{mov cx,17} pendant
le processus de compilation.

Il y a diff�rente fa�on de d�finir des labels. Le plus simple est
de mettre le nom du label suivi par (\verb":"), cette directive
peut �tre suivie par d'autres instructions sur la m�me ligne. Elle
donne une valeur qui est son adresse. Cette m�thode est utilis�e
pour d�finir un emplacement dans le code. L'autre fa�on est de
faire suivre le nom du label (sans les (\verb":")) par une
directive de donn�e. Cela d�fini le label avec une valeur �gale �
l'adresse/offset de d�but de d�finition des donn�es, et un label
de donn�e avec une taille comme il est sp�cifi� dans le tableau
1.3.

Le label peut �tre consid�r� comme une constante de valeur �gale �
l'offset du code ou de la donn�e fix�. Par exemple quand vous
d�finissez une donn�e utilisant cette directive \texttt{char db
224}, pour mettre l'addresse de cette valeur dans le registre
\texttt{bx} vous devrez utiliser l'instruction \texttt{mov
bx,char}, et pour mettre la valeur d'un octet adress� par le label
\texttt{char} dans le registre \texttt{dl}, vous devrez utiliser
\texttt{mov dl,[char]} (ou \texttt{mov dl,ptr char}). Mais si vous
essayez d'assembler \texttt{mov ax,[char]} cela donnera une
erreur, car fasm compare la taille des op�randes qui doit toujours
�tre �gale. Vous pouvez forcer l'assemblage de cette instruction
en utilisant un d�passement de taille: \texttt{mov ax,word
[char]}, mais rappelez-vous que cette instruction lira les deux
octets se trouvant � l'adresse \texttt{char} alors qu'il n'y �
qu'un octet d�fini.

La derni�re et la plus flexible des mani�res de d�finir un label
est d'utiliser la directive \texttt{label}. Cette directive doit
�tre suivi par le nom du label puis, en option, de l'op�rateur de
taille et, encore en option, de l'op�rateur \texttt{at} et de
l'expression num�rique indiquant l'adresse d'ou ce label devra
�tre d�fini. Par exemple \texttt{label wchar word at char}
d�finira un nouveau label (\texttt{wchar}) pour une donn�e de 16
bits (\texttt{word}) � l'adresse \texttt{char}. L'instruction
\texttt{mov ax,[wchar]} deviendra apr�s compilation la m�me chose
que \texttt{mov ax,word [char]}. Si aucune adresse n'est indiqu�e,
la directive \texttt{label} d�fini l'adresse du label � sa
position courante. Ainsi \texttt{mov [wchar],57568} copiera deux
octets tandis que \texttt{mov [char],224} copiera un seul octet �
la m�me adresse.

Le label qui a un nom commen�ant par un (\texttt{.}) sera
consid�r� comme un label local, et son nom est attach� au nom du
dernier label global (qui commence par n'importe quoi sauf un
"\texttt{.}") pour faire le nom complet du label. Donc vous pouvez
utilisez le nom court (commen�ant par un "\texttt{.}") de ce label
n'importe ou avant la d�finition du prochain label global, et dans
un autre endroit de votre source vous devez utilisez le nom
complet. Il y a une exception avec les labels commen�ant par deux
"\texttt{.}" - ils sont comme des labels globaux mais ne servent
pas de pr�fixes pour les labels locaux.

Le label anonyme est d�fini par \texttt{@@}, vous pouvez
l'utiliser plusieurs fois dans votre source. Le symbole
\texttt{@b} (et son �quivalent \texttt{@r}) fait r�f�rence au
pr�c�dent label anonyme, alors que le symbole \texttt{@r} fait
r�f�rence au label anonyme suivant. Ces symboles sp�ciaux ne font
pas de distinctions entre minuscule et majuscule.

\index{Introduction!Expressions num�riques}
\subsection{Expressions num�riques}
Dans les exemples que nous venons de voir, toutes les expressions
num�riques ne sont que de simples nombres, constantes ou labels.
Mais elles peuvent �tre plus complexes en utilisant des op�rateurs
arithm�tiques ou logiques pour des calculs au moment de la
compilation. Les priorit�s des ces op�rateurs se trouvent dans le
tableau 1.4. Les op�rations avec une priorit� �lev�e seront
calcul�es en premier, bien sur vous pouvez changer le comportement
du ces calculs en mettant certaines instructions
entre-parenth�ses. \texttt{+}, \texttt{-}, \texttt{*} et
\texttt{/} sont les op�rateurs arithm�tiques standards,
\texttt{mod} donne le reste d'une division. \texttt{and},
\texttt{or}, \texttt{xor}, \texttt{shl}, \texttt{shr} et
\texttt{not} ont la m�me fonction que les instructions assembleur
de m�me nom. \texttt{rva} permet la conversion d'une adresse en un
offset readressable et est sp�cifique � certain des formats de
sortie (voir section 2.4).
\begin{table}[!h]
    \begin{center}
        \begin{tabular}{|c|c|}
        \hline
        Priorit�&Op�rateurs\\
        \hline\hline
        0&\verb"+"\\
        &\verb"-"\\
        \hline
        1&\verb"*"\\
        &\verb"/"\\
        \hline
        2&\verb"mod"\\
        \hline
        3&\verb"and"\\
        &\verb"or"\\
        &\verb"xor"\\
        \hline
        4&\verb"shl"\\
        &\verb"shr"\\
        \hline
        5&\verb"not"\\
        \hline
        6&\verb"rva"\\
        \hline
        \end{tabular}
    \end{center}
    \caption{Op�rateurs arithm�tiques et logiques par priorit�}
\end{table}

Les nombre dans chaque expression est consid�r� comme un chiffre
d�cimal, les nombres binaires doivent avoir la lettre \texttt{b}
ajout� � la fin, un nombre octal se termine par la lettre
\texttt{o}, les nombres hexad�cimaux commencent avec \texttt{0x}
(comme avec le langage C) ou avec le caract�re \verb"$" (comme en
Pascal) ou bien se termine avec la lettre \texttt{h}. De m�me pour
les cha�nes, lorsqu'on les rencontre dans une expression, seront
converties en nombre - le premier caract�re deviendra l'octet le
moins significatif du nombre.

L'expression num�rique utilis�e comme une adresse peut aussi
contenir n'importe quels registres g�n�raux utilis�s pour
l'adressage, ils peuvent �tre additionn�s et multipli�s par des
valeurs appropri�es, comme il est autoris� pour les instructions
x86.

Il y a aussi quelques symboles sp�ciaux qui peuvent �tre utilis�s
� l'int�rieur de l'expression num�rique. Premi�rement il y a
\verb"$", qui est toujours �gale � la valeur de l'offset courant,
tandis que \verb"$$" est �gale � l'adresse de base de l'espace de
l'adresse courante. Puis il y a \verb"%", qui est le nombre de
r�p�tition courante dans les parties du code qui sont r�p�t�es en
utilisant quelques directives sp�ciales (voir section 2.2). Il y a
aussi le symbole \verb"%t", qui est toujours �gal � l'heure courante.

Toute expressions num�riques peuvent aussi �tre compos�es d'une
simple valeur en virgule flottante en notation scientifique (flat
assembler n'autorise pas toute op�ration en virgule flottante au
moment de la compilation), elles peuvent se terminer par le
caract�re \texttt{f} pour �tre reconnues, autrement elles doivent
contenir au moins le caract�re \verb"." ou le caract�re \verb"E".
Donc \texttt{1.0}, \texttt{1E0} et \texttt{1f} exprime la m�me
valeur en virgule flottante, alors qu'un simple \texttt{1} d�fini
une valeur enti�re.

\index{Introduction!Sauts et Appels}
\subsection{\label{subsec:SautEtAppels}Sauts et Appels}
L'op�rande de toute instruction de saut ou d'appel peut �tre
pr�c�d� pas seulement par l'op�rateur de taille, mais aussi par
les op�rateurs indiquant le type du saut: \texttt{short},
\texttt{near} ou \texttt{far} (\emph{ndt: respectivement : court,
proche ou loin}). Par exemple, lorsque l'assembleur est en mode 16
bits, l'instruction \texttt{jmp dword [0]} deviendra un saut
lointain (far: en dehors du segment courant) et quand l'assembleur
est en mode 32 bits, il deviendra un saut proche (near: dans le
segment courant). Pour forcer l'instruction a �tre trait�e
diff�rement, on utilise \texttt{jmp near dword [0]} ou \texttt{jmp
far dword [0]}.

Quand l'op�rande d'un saut proche (near jump) est une valeur
imm�diate, l'assembleur g�n�rera une variante de cette instruction
de saut plus courte, si cela lui est possible (mais ne cr�era pas
d'instruction 32 bits dans un mode 16 ni d'instruction 16 bits
dans un mode 32, � moins d'utiliser un op�rateur de taille avec).
En indiquant le type de saut vous pouvez le forcer � toujours
g�n�rer une variante longue (par exemple \texttt{jmp near 0}) ou
pour g�n�rer une variante courte qui se termine avec une erreur
lorsque cela devient impossible (par exemple \texttt{jmp short
0}).

\index{Introduction!R�glage de la taille}
\subsection{R�glage de la taille}
Quand une instruction utilise quelque adressage de m�moire, la
plus petite instruction est g�n�r� par d�faut en utilisant le
d�placement court si la valeur de l'adresse s'ajuste � la
distance. Cela peut �tre d�pass� en utilisant l'op�rateur
\texttt{word} ou \texttt{dword} avant l'adresse entre-crochets (ou
apr�s l'op�rateur \texttt{ptr}), qui force le long d�placement
d'une taille appropri�e � �tre faite. Au cas o� l'adresse n'est
relatif � aucun registre, ces op�rateurs autorisent de choisir le
mode appropri� d'adressage absolu.

Les instructions \texttt{adc}, \texttt{add}, \texttt{and},
\texttt{cmp}, \texttt{or}, \texttt{sbb}, \texttt{sub} et
\texttt{xor} avec un premier op�rande �tant en mode 16 ou 32 bits
sont, par d�faut, g�n�r�es sous la forme courte 8 bits quand le
deuxi�me op�rande est ajust� en une valeur sign� de 8 bits. Il
peut �tre augmenter en mettant l'op�rande \texttt{word} ou
\texttt{dword} avant la valeur imm�diate. Les r�gles similaires
s'appliquent � l'instruction \texttt{imul} avec la derni�re
op�rande �tant une valeur imm�diate.

Une valeur imm�diate comme op�rande pour l'instruction
\texttt{push} sans op�rateur de taille est consid�r�, par d�faut,
comme valeur "mot" (word), si l'assembleur est en mode 16 bits, et
en double mot (dword) si l'assembleur est en mode 32 bits, si cela
est possible on utilise la forme 8 bits de cette instruction,
l'op�rateur de taille \texttt{word} ou \texttt{dword} force
l'instruction \texttt{push} d'�tre g�n�r� sous une forme longue de
la taille sp�cifi�e. Les mn�moniques \texttt{pushw} et
\texttt{pushd} force l'assembleur a g�n�rer du code 16 ou 32 bits
sans le forcer � utiliser une forme d'instruction longue.
            % Chapitre Introduction
\chapter{Jeu d'instructions}
Ce chapitre fourni les informations d�taill�es sur les
instructions et les directives support�es par flat assembler. Les
directives pour d�finir les labels ont �t� vues dans la
sous-section 1.2.3, toutes les autres directives seront d�crites
plus tard dans ce chapitre.

\section{Les instructions de l'architecture x86}
Dans cette section vous trouverez les informations sur la syntaxe
et le but des instructions du langage assembleur. Si vous avez
besoin de plus d'informations techniques, veuillez vous reporter
au Manuel du D�veloppeur de Logiciel sous Architecture Intel sur
le site
\begin{verbatim}    http://www.intel.com/design/pentium4/manuals/index_new.htm
\end{verbatim}

Les instructions de l'assembleur se composent de mn�moniques (nom
de l'instruction) et de z�ro � trois op�randes. S'il y a deux
op�randes ou plus, le premier est la destination et le second est
la source. Chaque op�randes peut �tre un registre, une zone
m�moire ou une valeur imm�diate (voir section 1.2 pour avoir plus
de d�tails sur la syntaxe des op�randes). Dans cette section il y
a des exemples des diff�rentes combinaisons d'op�randes (si
l'instruction en est compos�e) apr�s leur description.

Quelques instructions se comportent comme un pr�fixe et peuvent
�tre suivies par d'autres instructions sur la m�me ligne, et il
peut y avoir plus d'un pr�fixe sur une ligne. Chaque nom de
registre de segment est aussi un mn�monique d'un pr�fixe
d'instruction, cependant il est recommand� d'utilis� un
d�passement de segment que l'on d�finit � l'int�rieur de crochets,
� la place de ces prefixes.

\subsection{D�placement de donn�es}
\texttt{mov}\emph{(destination,source)} transfert un octet, mot ou
double-mot de l'op�rande source vers l'op�rande destination, de
cette fa�on :
\begin{verbatim}    mov destination,source
\end{verbatim}
Il peut transf�rer des donn�es entre des registres, d'un registre
vers une zone m�moire, ou d'une zone m�moire vers un registre,
mais il ne peut pas faire de d�placements de zone m�moire � zone
m�moire. Il peut aussi transf�rer une valeur imm�diate vers une
zone m�moire ou une registre, d'un registre de segment vers un
registre g�n�ral ou une zone m�moire, registre g�n�ral ou zone
m�moire vers un registre de segment, d'un registre de contr�le ou
de debug vers un registre g�n�ral et r�ciproquement. L'assemblage
de \texttt{mov} ne pourra s'effectuer que si la taille de
l'op�rande source et destination sont identiques. Voici les
exemples de toutes les combinaisons possibles :
\begin{verbatim}    mov bx,ax           ; registre g�n�ral vers registre g�n�ral
    mov [char],al       ; registre g�n�ral vers zone m�moire
    mov bl,[char]       ; zone m�moire vers registre g�n�ral
    mov dl,32           ; Valeur imm�diate vers registre g�n�ral
    mov [char],32       ; Valeur imm�diate vers zone m�moire
    mov ax,ds           ; registre de segment vers registre g�n�ral
    mov [bx],ds         ; registre de segment vers zone m�moire
    mov ds,ax           ; registre g�n�ral vers registre de segment
    mov ds,[bx]         ; zone m�moire vers registre de segment
    mov eax,cr0         ; registre de contr�le vers registre g�n�ral
    mov cr3,ebx         ; registre g�n�ral vers registre de contr�le
\end{verbatim}

\texttt{xchg}\emph{(destination,source)} �change le contenu de
deux op�randes. Il peut �changer deux op�randes de type octets,
deux mots ou deux double-mot. L'ordre des op�randes n'est pas
important. Les op�randes peuvent �tre des registres g�n�raux ou
des zones m�moires. Par exemple :
\begin{verbatim}    xchg ax,bx          ; �change deux registres g�n�raux
    xchg al,[char]      ; �change un registre avec une zone m�moire
\end{verbatim}

\texttt{push} \emph{destination} d�cr�mente l'indicateur du
pointeur de pile (registre \texttt{esp}) et transfert l'op�rande
en haut de la pile d�sign� par \texttt{esp}. L'op�rande peut �tre
une zone m�moire, un registre g�n�ral, un registre de segment ou
une valeur imm�diate de type octet, mot ou double-mot. Si
l'op�rande est une valeur imm�diate et qu'aucune taille n'est
indiqu�e alors il est consid�r�, par d�faut, comme un mot (word)
si l'assembleur est en mode 16 bits et en double-mot sir
l'assembleur est en mode 32 bits. Les mn�moniques \texttt{pushw}
et \texttt{pushd} sont les variantes de l'instruction qui stocke
les valeurs d'un mot ou d'un double-mot. Si il y plusieurs
op�randes sur la m�me ligne (s�par�es par des espaces et pas par
des virgules), le compilateur assemblera la suite de ces
instructions. Exemple avec des op�randes uniques :
\begin{verbatim}    push ax             ; Stocke un registre g�n�ral
    push es             ; Stocke une registre de segment
    push [bx]           ; Stocke une zone m�moire
    push 1000h          ; Stocke une valeur imm�diate

    push ecx edx eax    ; Stocke plusieurs registres g�n�raux
\end{verbatim}

\texttt{pusha} sauvegarde le contenu des huit registres g�n�raux
sur la pile. Cette instruction n'a aucune op�rande. Il y a deux
versions de cette instruction, une version 16 bits et un version
32 bits, l'assembleur g�n�re automatiquement la bonne version pour
le mode courant, mais il peut �tre d�pass� en utilisant les
mn�moniques \texttt{pushaw} ou \texttt{pushad} pour toujours avoir
la version 16 ou 32 bits. La version 16 bits empile les registres
g�n�raux sur la pile suivant cet ordre : \texttt{ax}, \texttt{cx},
\texttt{dx}, \texttt{bx}, la valeur de \texttt{sp} avant que
\texttt{ax} soit empil�, \texttt{bp}, \texttt{si} et \texttt{di}.
M�me chose pour la version 32 bits. Si on repr�sente cette
instruction comme un programme, nous avons :
\begin{verbatim}    IF Taille_Operand = 16   ; instruction PUSHA
       Temp = (SP)
       Push(AX)
       Push(CX)
       Push(DX)
       Push(BX)
       Push(Temp)
       Push(BP)
       Push(SI)
       Push(DI)
    ELSE                     ; Taille_Operande = 32 => instruction PUSHAD
       Temp = (ESP)
       Push(EAX)
       Push(ECX)
       Push(EDX)
       Push(EBX)
       Push(Temp)
       Push(EBP)
       Push(ESI)
       Push(EDI)
    ENDIF
\end{verbatim}

\texttt{pop} \emph{destination} transf�re le mot ou double-mot du
haut de la pile vers l'op�rande de destination puis incr�mente
\texttt{esp} pour pointer vers la nouvelle valeur du haut de la
pile. L'op�rande peut �tre une zone m�moire, un registre g�n�ral
ou un registre de segment. Les mn�moniques \texttt{popw} et
\texttt{popd} sont les variantes de l'instruction qui r�cup�re les
valeurs d'un mot ou d'un double-mot. Si plusieurs op�randes sont
sur la m�me ligne et s�par�es par des espaces, l'assembleur
compilera la suite des instructions.
\begin{verbatim}    pop bx              ; Restaure un registre g�n�ral
    pop ds              ; Restaure le segment de registre
    popw [si]           ; Restaure une zone m�moire

    pop eax edx ecx     ; Restaure plusieurs registres g�n�raux
                        ; Notez l'ordre des registres empil�s par le push
                        ; et l'ordre des registres d�pil�s par le pop
\end{verbatim}

\texttt{popa} restaure les registres sauvegard�s sur la pile par
l'instruction \texttt{pusha} � part la valeur de \texttt{sp} (ou
\texttt{esp}), qui est ignor�e. Cette instruction n'a pas
d'op�randes. Pour forcer l'assemblage de cette instruction en mode
16 ou 32 bits, il faut utiliser les mn�moniques \texttt{popaw} ou
\texttt{popad}.

\subsection{Conversion de type}
Les instructions de conversions de types convertissent les octets
en mots, mots en doubles-mot et doubles-mot en quadruples-mot. Ces
conversions peuvent se faire en ajoutant une extension de signe ou
une extension par z�ro. L'extension de signe remplit les bits
suppl�mentaires du plus grand �l�ment avec la valeur du bit de
signe du plus petit �l�ment, l'extension de z�ro remplit les bits
suppl�mentaires par des z�ros.

\texttt{cwd} et \texttt{cdq} doublent la taille des valeurs des
registres \texttt{ax} et \texttt{eax} et sauvegardent les bits
suppl�mentaires dans les registres \texttt{dx} ou \texttt{edx}. La
conversion s'effectue en utilisant l'extension de signe. Ces
instructions n'ont pas d'op�randes.

\texttt{cbw} �tend le signe de l'octet du registre \texttt{al}
faisant partie de \texttt{ax}, et \texttt{cwde} �tend le signe du
mot \texttt{ax} faisant partie de \texttt{eax}. Ces instructions
n'ont pas d'op�randes.

\texttt{movsx} converti un octet en mot ou double-mot et un mot en
double-mot avec l'extension de signe. \texttt{movzx} fait la m�me
chose, mais il utilise l'extension par z�ro. L'op�rande source
peut �tre un registre g�n�ral ou une zone m�moire, alors que
l'op�rande de destination doit �tre un registre g�n�ral. Par
exemple :
\begin{verbatim}    movsx ax,al         ; registre octet vers registre mot
    movsx edx,dl        ; registre octet vers registre double-mot
    movsx eax,ax        ; registre mot vers registre double-mot
    movsx ax,byte [bx]  ; zone m�moire octet vers registre mot
    movsx edx,byte [bx] ; zone m�moire octet vers registre double-mot
    movsx eax,word [bx] ; zone m�moire mot vers registre double-mot
\end{verbatim}

\subsection{Les instructions de l'arithm�tique binaire}
\texttt{add}\emph{(destination,source)} remplace l'op�rande de
destination avec la somme des op�randes source et destination et
met l'indicateur \texttt{CF} (Carry Flag : le drapeau de retenue)
� 1 si un d�bordement se produit. Les op�randes peuvent �tre des
octets, mots ou doubles-mot. L'op�rande destination peut �tre un
registre g�n�ral ou une zone m�moire, l'op�rande source un
registre g�n�ral ou une valeur imm�diate, il peut aussi �tre une
zone m�moire si la destination est un registre.
\begin{verbatim}    add ax,bx           ; Additionne un registre � un registre
    add ax,[si]         ; Additionne une zone m�moire � un registre
    add [di],al         ; Additionne un registre � une zone m�moire
    add al,48           ; Additionne une valeur imm�diate � un registre
    add [char],48       ; Additionne une valeur imm�diate � une zone m�moire
\end{verbatim}

\texttt{adc}\emph{(destination,source)} fait la somme des
op�randes, ajoute 1 si l'indicateur \texttt{CF} est positionn�, et
remplace l'op�rande de destination par le r�sultat. Cette
instruction suit les m�mes r�gles que l'instruction \texttt{add}.
\texttt{add} suivit par plusieurs instructions \texttt{adc} peut
�tre utilis� pour additionner des nombres sup�rieurs � 32 bits.

\texttt{inc} \emph{destination} ajoute 1 � l'op�rande, il
n'affecte pas \texttt{CF}. L'op�rande peut �tre un registre
g�n�ral ou une zone m�moire, et la taille de celle-ci peut �tre un
octet, un mot ou un double-mot.
\begin{verbatim}    inc ax              ; Incr�mente le registre par 1
    inc byte [bx]       ; Incr�mente la zone m�moire par 1
\end{verbatim}

\texttt{sub}\emph{(destination,source)} soustrait l'op�rande
source de l'op�rande destination et remplace la destination par le
r�sultat. L'indicateur de retenue \texttt{CF} est pos� si le
r�sultat ne tient pas dans la destination. Cette instruction suit
les m�mes r�gles que l'instruction \texttt{add}.

\texttt{sbb}\emph{(destination,source)} soustrait l'op�rande
source de l'op�rande destination et soustrait 1 si l'indicateur
\texttt{CF} est pos�, et stocke le r�sultat dans l'op�rande de
destination. Cette instruction suit les m�mes r�gles que
l'instruction \texttt{add}. \texttt{sub} suivit par plusieurs
instructions \texttt{sbb} peut �tre utilis� pour soustraire des
nombres plus grand que 32 bits.

\texttt{dec} \emph{destination} soustrait 1 � l'op�rande, il
n'affecte pas l'indicateur \texttt{CF}. Cette instruction suit les
m�me r�gles que l'instruction \texttt{inc}.

\texttt{cmp}\emph{(destination,source)} soustrait l'op�rande
source de l'op�rande destination. Il met � jour les indicateurs
comme pour l'instruction \texttt{sub}, mais ne modifie pas la
source et la destination. Cette instruction suit les m�me r�gles
que l'instruction \texttt{sub}.

\texttt{neg} \emph{destination} soustrait un entier sign� de z�ro.
L'effet de cette instruction est de transformer le signe de
l'op�rande de n�gatif � positif ou inversement. Cette instruction
suit les m�me r�gles que l'instruction \texttt{inc}.

\texttt{xadd}\emph{(destination,source)} �change l'op�rande de
destination avec l'op�rande source, puis il charge la somme de ces
deux valeurs dans l'op�rande de destination. Cette instruction
suit les m�me r�gles que l'instruction \texttt{add}. Exemple :
\begin{verbatim}    mov bx,1200         ; bx = 1200
    mov dx,2500         ; dx = 2500
    xadd bx,dx          ; bx = 3700 - dx = 1200
\end{verbatim}

Toutes ces instructions arithm�tiques binaires modifient les
indicateurs \texttt{SF} \emph{(Sign Flag : indicateur de signe -
bit 7)}, \texttt{ZF} \emph{(Zero Flag : indicateur de z�ro - bit
6)}, \texttt{PF} \emph{(Parity Flag : indicateur de parit� - bit
2)} et \texttt{OF} \emph{(Overflow Flag : indicateur de
d�bordement - bit 11)}. L'indicateur \texttt{SF} a la m�me valeur
que le r�sultat du bit de signe, \texttt{ZF} est mis � 1 quand le
r�sultat d'une op�ration donne 0 (z�ro), \texttt{PF} est mis � 1
quand les 8 bits de poids faibles du r�sultat d'une op�ration
contient un nombre pair de bits �gaux � 1, \texttt{OF} est mis � 1
si le r�sultat est trop grand pour un nombre positif ou trop petit
pour un nombre positif (bit de signe exclu).

\texttt{mul} \emph{source} ex�cute une multiplication non-sign�e
de l'op�rande source et de l'accumulateur. Si l'op�rande est un
octet, le processeur le multiplie par le contenu du registre
\texttt{al} et retourne le r�sultat au format 16 bits dans
\texttt{ah} et \texttt{al}. Si l'op�rande est un mot, le
processeur le multiplie le contenu de \texttt{ax} et retourne le
r�sultat au format 32 bits dans les registres \texttt{dx} et
\texttt{ax}. Si l'op�rande est un double-mot, le processeur le
multiplie par le contenu de \texttt{eax} et retourne le r�sultat
au format 64 bits dans les registres \texttt{edx} et \texttt{eax}.
\texttt{mul} met � 1 les indicateurs \texttt{CF} et \texttt{OF}
quand la moiti� haute du r�sultat est diff�rent de 0, autrement
ces indicateurs sont mis � 0. Cette instruction suit les m�me
r�gles que l'instruction \texttt{inc}.

\texttt{imul} effectue une multiplication sign�e. Cette
instruction a trois variantes. La premi�re n'a qu'une seule
op�rande (\texttt{imul} \emph{source}) et se comporte de la m�me
fa�on que l'instruction \texttt{mul}. La seconde a deux op�randes
(\texttt{imul}\emph{(destination,source)}), l'op�rande destination
est multipli� � l'op�rande source et le r�sultat est mis dans la
destination. L'op�rande destination doit �tre un registre g�n�ral,
un mot ou un double-mot, l'op�rande source est un registre
g�n�ral, une zone m�moire ou une valeur imm�diate. La troisi�me
variante a trois op�randes (\texttt{imul}\emph{(destination,
source1, source2)}), l'op�rande destination doit �tre un registre
g�n�ral, un mot ou un double-mot, l'op�rande source1 est un
registre g�n�ral ou une zone m�moire, et le dernier op�rande
source2 doit �tre une valeur imm�diate. L'op�rande source1 est
multipli� par l'op�rande source2 (une valeur imm�diate) et le
r�sultat est stock� dans l'op�rande destination. Ces trois formes
calculent le produit de la taille des op�randes par deux et
mettent \texttt{CF} et \texttt{OF} � 1 quand la moiti� haute du
r�sultat est diff�rent de z�ro, mais la seconde et troisi�me
variante de cette instruction tronque le produit de la taille des
op�randes, ces variantes peuvent donc aussi �tre utilis�es pour
des op�randes non-sign�es car, que les op�randes soient sign�es ou
non, la moiti� basse du produit est identique. Voici des exemples
de ces trois variantes :
\begin{verbatim}    imul bl         ; Accumulateur par registre
    imul word [si]  ; Accumulateur par zone m�moire
    imul bx,cx      ; Registre par registre
    imul bx,[si]    ; Registre par zone m�moire
    imul bx,10      ; Registre par valeur imm�diate
    imul ax,bx,10   ; bx * 10 => registre ax
    imul ax,[si],10 ; Zone m�moire par valeur imm�diate vers registre
\end{verbatim}

\texttt{div} \emph{source} effectue une division non-sign�e de
l'accumulateur par l'op�rande. Le dividende (accumulateur) est
deux fois la taille du diviseur (op�rande source), le quotient et
le reste de la division ont la m�me taille que le diviseur. Si le
diviseur est un octet, le dividende sera alors le registre
\texttt{ax}, le quotient est stock� dans \texttt{al} et le reste
dans \texttt{ah}. Si le diviser est un mot, la moiti� haute du
dividende est pris du registre \texttt{dx}, la moiti� basse du
dividende est pris du registre \texttt{ax}, le quotient est stock�
dans \texttt{ax} et le reste dans \texttt{dx}. Si le diviseur est
une double-mot, la moiti� haute du dividende est pris du registre
\texttt{edx}, la partie basse du dividende est pris du registre
\texttt{eax}, le quotient est stock� dans \texttt{eax} et le reste
dans \texttt{edx}. Cette instruction suit les m�me r�gles que
l'instruction \texttt{mul}.

\texttt{idiv} \emph{source} effectue une division sign�e de
l'accumulateur par l'op�rande. Il utilise les m�mes registres que
l'instruction \texttt{div} et suit les m�mes r�gles que cette
op�rande.

\subsection{Les instructions de l'arithm�tique d�cimale}
L'arithm�tique d�cimal s'effectue en combinant l'arithm�tique
binaire (d�crite dans la section pr�c�dente) avec les instructions
de l'arithm�tique d�cimal. Ces instructions sont utilis�es pour
ajuster une pr�c�dente op�ration arithm�tique binaire pour
produire un r�sultat d�cimal compact� ou non compact� valide, ceci
afin d'ajuster les prochaines entr�es d'une op�ration arithm�tique
binaire, l'op�ration produira un r�sultat d�cimal compact� ou non
compact� valide.

\texttt{daa} ajuste en \texttt{al} le r�sultat d'une addition de
deux nombres d�cimaux compact�s, \texttt{daa} doit toujours suivre
l'addition de deux nombres d�cimaux compact�s (un chiffre pour
chaque demi-octet) pour obtenir comme r�sultat deux nombre
d�cimaux compact�s valides. L'indicateur de retenue est mis � 0 si
le contenu est correct (entre 0 et 99 car les chiffres sont
repr�sent�s en BCD \emph{Binary Coded Decimal - D�cimal Cod� en
Binaire}). Cette instruction est sans op�rande.

\texttt{das} ajuste en \texttt{al} le r�sultat d'une soustraction
de deux nombres d�cimaux compact�s, \texttt{das} doit toujours
suivre la soustraction de deux nombres d�cimaux compact�s (un
chiffre pour chaque demi-octet) pour obtenir comme r�sultat deux
nombre d�cimaux compact�s valides. L'indicateur de retenue
\texttt{CF} est mis � 1 s'il y a d�passement (> 99). Cette
instruction est sans op�rande.

\texttt{aaa} change le contenu du registre \texttt{al} en un
nombre d�cimal non compact� valide, et rempli de z�ros les quatre
bits (ou quartet) de poids fort. \texttt{aaa} doit toujours suivre
l'addition des deux op�randes d�cimales non compact�es dans
\texttt{al}. L'indicateur de retenue est pos� et \texttt{ah} est
incr�ment� si une retenue est n�cessaire. Cette instruction est
sans op�rande. Exemple :
\begin{verbatim}    mov ah,8
    mov al,4
    add al,ah       ; al = al + ah --> al = $0C
    xor ah,ah       ; ah = 0
                    ; on ajuste au format BCD
    aaa             ; ah = 1 et al = 2
\end{verbatim}

\texttt{aas} change le contenu du registre \texttt{al} en un
nombre d�cimal non compact� valide, et rempli de z�ros les quatre
bits (ou quartet) de poids fort. \texttt{aas} doit toujours suivre
la soustraction des deux op�randes d�cimales non compact�es dans
\texttt{al}. Si la soustraction produit une retenue, le registre
\texttt{ah} est incr�ment�, et les indicateurs \texttt{CF} et
\texttt{AF} \emph{(AF : Auxiliary Flag - Indicateur de retenue
auxiliaire)} sont mis � 1. Dans le cas contraire \texttt{CF} et
\texttt{AF} sont mis � 0 et \texttt{ah} reste inchang�. Cette
instruction est sans op�rande.

\texttt{aam} corrige le r�sultat d'une multiplication de deux
nombres d�cimaux non compact�s valides. \texttt{aam} doit toujours
suivre une multiplication de deux nombres d�cimaux pour produire
un r�sultat d�cimal valide. Le chiffre de poids fort est stock�
dans \texttt{ah}, le chiffre de poids faible dans \texttt{al}. La
version g�n�ralis�e de cette instruction permet � l'ajustement du
contenu du registre \texttt{ax} de cr�er deux chiffres non
compact�es. \texttt{aam} d�compacte le r�sultat dans \texttt{al}
en divisant al par 10, laissant le quotient (chiffre de poids
fort) dans \texttt{ah} et le reste (chiffre de poids faible) dans
\texttt{al}. Cette instruction n'a pas d'op�randes. Exemple :
\begin{verbatim}    mov al,8
    mov cl,4
    mul cl          ; ax = al * cl --> ax = 32 ($20)
                    ; on ajuste le r�sultat
    aam             ; ah = 3 et al = 2
\end{verbatim}

\texttt{aad} pr�pare deux valeurs d�cimales non compact�s � �tre
trait� par une division et modifie le num�rateur dans \texttt{al}
et \texttt{ah}, ainsi le quotient produit par la division sera un
nombre d�cimal non compact�. \texttt{ah} contient le chiffre de
poids fort et \texttt{al} le chiffre de poids faible. Cette
instruction ajuste la valeur et met le r�sultat dans \texttt{al}
tandis que \texttt{ah} sera �gal � z�ro. Cette instruction suit
les m�me r�gles que l'instruction \texttt{aam}. Exemple :
\begin{verbatim}    mov al,8
    mov ah,4
    aad             ; pr�pare ax � une division
                    ; apr�s cette instruction al = 48 et ah = 0
\end{verbatim}

\subsection{Instructions logiques}
L'instruction \texttt{not} inverse les bits de l'op�rande, ce qui
permet de faire le compl�ment � un de cette op�rande. Il n'a aucun
effet sur les indicateurs (flags). Cette instruction suit les m�me
r�gles que l'instruction \texttt{inc}.

Les instructions \texttt{and}, \texttt{or} et \texttt{xor}
ex�cutent les op�rations standards logiques. Ils modifient les
indicateurs \texttt{SF} \emph{(Indicateur de Signe-Bit 7)},
\texttt{ZF} \emph{(Indicateur de Z�ro-Bit 6)} et \texttt{PF}
\emph{(Indicateur de Parit�-Bit 2)}. Cette instruction suit les
m�me r�gles que l'instruction \texttt{add}.

Les instructions \texttt{bt}\emph{(destination,source)},
\texttt{bts}, \texttt{btr} et \texttt{btc} op�rent sur un seul bit
d'une zone m�moire ou d'un registre g�n�ral. Le rang du bit �
tester est indiqu� par la position du bit de l'op�rande source
(deuxi�me op�rande), qui peut �tre une valeur imm�diate ou un
registre. Ces instructions extrait le bit de l'op�rande
destination et le met dans l'indicateur \texttt{CF}. L'instruction
\texttt{bt} ne fait rien d'autre, \texttt{bts} met le bit
s�lectionn� � 1, \texttt{btr} fait l'inverse, il met le bit
s�lectionn� � 0 et \texttt{btc} inverse le bit s�lectionn�. Le
premier op�rande peut �tre un mot ou un double-mot. Exemple :
\begin{verbatim}    bt  ax,15           ; Teste le bit 15 du registre ax
    bts word [bx],15    ; Test et met le bit 15 de la m�moire � 1
    btr ax,cx           ; Test et met le bit de cx dans ax � 0
    btc word [bx],cx    ; Test et inverse le bit de cx dans la m�moire

    mov ax,1000000000000000b    ; ax = 32768
    bt  ax,1000000000000000b    ; test le bit 15 de ax
\end{verbatim}

Les instructions \texttt{bsf}\emph{(destination,source)} et
\texttt{bsr} recherchent le premier bit d'un mot ou d'un
double-mot et stocke l'index de ce bit dans l'op�rande destination
qui est un registre g�n�ral. Le bit recherch� est indiqu� par
l'op�rande source qui est un registre ou une zone m�moire. Si la
recherche aboutit l'indicateur \texttt{ZF} est mis � 1 et
l'op�rande source contient l'index du bit. Si aucun bit n'est
trouv�, la source et l'indicateur \texttt{ZF} sont mis � 0.
\texttt{bsf} fait une recherche en partant du bit 0 et en
remontant vers le bit de poids fort. \texttt{bsr} fait une
recherche en partant du bit de poids fort (bit 31 pour un
double-mot, bit 15 pour un mot) et redescend vers le bit 0.
\begin{verbatim}    bsf ax,bx           ; Recherche 1er bit � 1 dans ax
                        ; met l'index dans bx en partant du bit 0
    bsr ax,[si]         ; Recherche le 1er bit � 1 dans ax
                        ; met l'index dans zone m�moire [si] en
                        ; partant du bit 15 vers bit 0
\end{verbatim}

\texttt{shl}\emph{(destination,source)} d�cale vers la gauche
l'op�rande destination par le nombre de bits indiqu� par
l'op�rande source. L'op�rande destination peut �tre un octet, un
mot, un double mot d'un registre g�n�ral ou une zone m�moire.
L'op�rande source peut �tre une valeur imm�diate ou le registre
\texttt{cl}. \texttt{shl} rempli de z�ro � partir de la droite de
l'op�rande (poids faible) et les bits disparaissent � gauche. Le
dernier bit d�cal� vers la gauche est stock� dans l'indicateur
\texttt{CF}. L'instruction \texttt{sal} est identique �
l'instruction \texttt{shl}.
\begin{verbatim}    shl al,1            ; D�cale le registre al d'un bit vers la gauche
                        ; => Correspond � une multiplication par 2
    shl byte [bx],1     ; D�cale la zone m�moire d'un bit vers la gauche
    shl ax,cl           ; D�cale vers la gauche le registre ax
                        ; du nombre de bits indiqu� dans cl
    shl word [bx],cl    ; D�cale vers la gauche la zone m�moire
                        ; du nombre de bits indiqu� dans cl
\end{verbatim}

\texttt{shr}\emph{(destination,source)}, et \texttt{sar}, d�cale
vers la droite la destination par le nombre de bits indiqu� par la
source. Cette instruction suit les m�me r�gles que l'instruction
\texttt{shl}. \texttt{shr} rempli de z�ro � partir de la gauche de
l'op�rande et les bits disparaissent � droite. Le dernier bit
d�cal� vers la droite est stock� dans l'indicateur \texttt{CF}.
Par contre l'instruction \texttt{sar} pr�serve le signe de
l'op�rande en le d�calant par des z�ros si la valeur est positive
et par des uns si la valeur est n�gative.
\begin{verbatim}    shr al,1            ; D�cale le registre al d'un bit vers la droite
                        ; => Correspond � une division par 2
    shr al,2            ; D�cale le registre al de deux bits vers la droite
                        ; => Correspond � une division par 4
\end{verbatim}

% Ajout du 01/12/2010
\texttt{shld}\emph{(destination,source,nombre)} d�cale vers la gauche
les bits de l'op�rande source du nombre de positions donn� par le
troisi�me op�rande, le r�sultat est mis dans la destination sans
modification de l'op�rande source. La destination peut �tre de
type Mot ou Double-Mot, un registre g�n�ral ou une zone m�moire,
la source doit �tre de type registre g�n�ral et le troisi�me op�rande
peut �tre une valeur imm�diate ou le registre \texttt{cl}.
\begin{verbatim}    shld ax,bx,1        ; d�cale le registre bx de 1 bit et le met dans ax
    shld [di],bx,1      ; d�cale bx de 1 bit
                        ; et met le r�sultat dans la zone m�moire [di]
    shld ax,bx,cl       ; d�cale bx du nombre donn� par cl :
                        ; r�sultat dans ax
    shld [di],bx,cl     ; d�cale bx du nombre donn� par cl
                        ; et met le r�sultat dans [di]
\end{verbatim}

\texttt{shrd}\emph{(destination,source,nombre)} d�cale vers la droite
les bits de l'op�rande source du nombre de positions donn� par l'op�rande
nombre, le r�sultat est mis dans la destination. L'op�rande source
n'est pas modifi�. Les r�gles sont les m�mes que l'instruction \texttt{shld}.

\texttt{rol} et \texttt{rcl}\emph{(destination,source)} d�cale, l'Octet,
Mot ou Double-Mot, vers la gauche les bits de la destination par le nombre
indiqu� par l'op�rande source. Pour \texttt{rol} : � chaque rotation le bit
de poids fort (bit le plus � gauche) qui sort par la gauche est plac� �
droite dans le bit de poids faible (bit '0').
\texttt{rcl} effectue la m�me chose sauf que le premier bit est plac� dans
l'indicateur \texttt{CF} apr�s que le contenu de cet indicateur ait �t� plac�
dans le bit '0' de la destination. Ces instructions suivent les m�mes r�gles
que l'instruction \texttt{shl}.

\texttt{ror} et \texttt{rcr}\emph{(destination,source)} d�cale, l'Octet,
Mot ou Double-Mot, vers la droite les bits de la destination par le nombre
indiqu� par l'op�rande source. Pour \texttt{ror} : � chaque rotation le bit
de poids faible (bit le plus � droite) qui sort par la droite est plac� �
gauche dans le bit de poids fort.
\texttt{rcr} effectue la m�me chose sauf que le dernier bit est plac� dans
l'indicateur \texttt{CF} apr�s que le contenu de cet indicateur ait �t� plac�
dans le bit de poids fort de la destination. Ces instructions suivent les m�mes
r�gles que l'instruction \texttt{shl}.

\texttt{test}\emph{(destination,source)} agit de la m�me fa�on que
l'instruction \texttt{and}, � la seule diff�rence qu'il n'alt�re pas la
l'op�rande de destination, il met juste les flags � jour.
Cette instruction suit la m�me r�gle que l'instruction \texttt{and}.

\texttt{bswap}\emph{(destination)} permute l'ordre des octets d'un registre
g�n�ral de 32 bits. Il permute les bits 0 � 7 avec les bits 24 � 31 et les
bits 8 � 15 avec les bits 16 � 23. Cette instruction est mise � disposition
pour la conversion des valeurs de type "little-endian" vers un format de type
"big-endian" et r�ciproquement.
\begin{verbatim}    bswap edx           ; permute les octets du registre edx
\end{verbatim}

\subsection{Les instructions de contr�le de transfert}
\texttt{jmp}\emph{(destination)} transfert le contr�le, d'une
fa�on inconditionnelle, � la destination. L'adresse de
destination peut �tre sp�cifi�e directement dans l'instruction
ou indirectement par un registre ou une zone m�moire, la taille de
cette adresse d�pend si le saut est proche ou lointain
\emph{(appel� aussi appel inter-segment)}(on peut le d�finir en
ajoutant les op�rateurs \texttt{'near'} ou \texttt{'far'} avant
l'op�rande) et si l'instruction est en 16 ou 32 bits. L'op�rande a
utiliser pour les sauts \texttt{'near'} doit �tre \texttt{word}
pour les instructions 16 bits et \texttt{dword} pour les instructions
32 bits. Pour les sauts \texttt{'far'}, on utilise \texttt{dword} pour
les instructions en 16 bits ou \texttt{pword} pour les instructions
32 bits. Une instruction directe \texttt{jmp} a obligatoirement
l'adresse de destination (on peut la pr�c�der de l'op�rateur \texttt{short},
\texttt{near} ou \texttt{far}), l'op�rande indiquant l'adresse doit
�tre une expression num�rique pour les sauts de type \texttt{near}
ou \texttt{short}, ou deux expressions num�riques s�par�es par \verb"':'"
pour les sauts \texttt{far}, la premi�re expression indique le segment et
le deuxi�me est l'offset � l'int�rieur de ce segment. L'op�rateur
\texttt{pword} peut �tre utilis� pour forcer un saut de type \texttt{far}
32 bits, et \texttt{dword} pour un saut \texttt{far} 16 bits.
Une instruction indirecte \texttt{jmp} nous donne l'adresse de destination
indirectement par un registre ou une variable pointeur, l'op�rande utilis�
doit �tre un registre ou une zone m�moire. Voir section~\ref{subsec:SautEtAppels}
pour plus de d�tails.
\begin{verbatim}    jmp 100h            ; saut direct proche (near)
    jmp 0FFFFh:0        ; saut direct lointain (far)
    jmp ax              ; saut indirect proche
    jmp pword [ebx]     ; saut indirect lointain
\end{verbatim}

\texttt{call}\emph{(destination)} appelle une proc�dure ou
sous-programme, l'adresse de l'instruction suivant le \texttt{call}
est sauvegard�e sur la pile, l'adresse de retour sera alors lue
lors de l'utilisation de l'instruction \texttt{ret} (return).
Cette instruction suit la m�me r�gle que l'instruction \texttt{jmp},
sauf que \texttt{call} n'a aucune version courte pour une instruction
directe et n'est donc pas optimis�.

\texttt{ret}, \texttt{retn} et \texttt{retf} mettent fin au
sous-programme et redonnent le contr�le au programme appelant
en utilisant l'adresse sauvegard�e sur la pile par l'instruction
\texttt{call}. \texttt{ret} est identique � \texttt{retn}, elle revient
de la proc�dure qui a �t� execut�e par un \texttt{call} de type
\texttt{'near'}, alors que \texttt{retf} revient d'un call de type
\texttt{'far'}. Ces instructions ont une taille par d�faut en fonction
des param�tres du code, la taille de l'adresse peut �tre forc�e � 16 bits
en utilisant les instructions \texttt{retw}, \texttt{retnw} et \texttt{retfw},
et � 32 bits par l'utilisation des instructions \texttt{retd},
\texttt{retnd} et \texttt{retfd}. En option, on peut ajouter une op�rande
imm�diate � ces instructions, en ajoutant cette constante au pointeur de la
pile, tous les arguments mis sur la pile, que le programme appelant a pass�,
sont effac�s avant l'execution de l'instruction \texttt{call}.

\texttt{iret} redonne le contr�le d'une proc�dure d'interruption.
Elle diff�re de \texttt{ret} car elle retourne aussi les 'flags'
de la pile dans le registre des flags. Ces flags sont sauvegard�s
sur la pile par le m�canisme de l'interruption. La taille de l'adresse
de retour est d�finie en fonction des param�tres du code,
mais peut-�tre forc�e � 16 ou 32 bits par l'utilisation des instructions
\texttt{iretw} ou \texttt{iretd}.

Les instructions de transferts conditionnels sont des sauts qui
transf�rent ou non le contr�le en fonction du statut des flags du CPU
quand l'instruction s'ex�cute. Les mn�moniques pour les sauts
conditionnels peuvent �tre obtenus en ajoutant le mn�monique de condition
ad�quat (voir tableau 2.1) au mn�monique \texttt{'j'}, par exemple
l'instruction \texttt{jc} transfert le contr�le si le flag \texttt{CF}
est d�fini. Les sauts conditionnels peuvent �tre de type \texttt{short}
ou \texttt{near}, uniquement directs, et peuvent �tre optimis�s
(voir~\ref{subsec:SautEtAppels}), l'op�rande doit �tre une valeur imm�diate
indiquant l'adresse de destination.

Les instructions \texttt{loop}\emph{(destination)} sont des sauts
conditionnels utilisant une valeur plac�e dans \texttt{cx} (ou \texttt{ecx}),
celle-ci donne le nombre de r�p�tition � effectuer dans la boucle du programme.
Toutes les instructions \texttt{loop} d�cr�mentent automatiquement
la valeur du registre \texttt{cx} (ou \texttt{ecx}) et termine la
boucle (sans donner le contr�le) quand \texttt{cx} (ou \texttt{ecx})
est �gal � z�ro. Ils utilisent \texttt{cx} ou \texttt{ecx} si les param�tres
du code actuel est en 16 ou 32 bits, mais on peut forcer l'utilisation de
\texttt{cx} avec l'instruction \texttt{loopw} \emph{(cx : 16 bits et
loopw : word)} ou l'utilisation de \texttt{ecx} avec \texttt{loopd}.
\texttt{loope} et \texttt{loopz} sont similaires � l'instruction \texttt{loop}
mais termine aussi la boucle quand le flag \texttt{ZF} est d�fini.
\texttt{loopew} et \texttt{loopzw} forcent l'utilisation du registre
\texttt{cx} alors que \texttt{looped} et \texttt{loopzd} force l'utilisation
de \texttt{ecx}. \texttt{loopne} et \texttt{loopnz} sont identiques �
l'instruction \texttt{loop} sauf que la boucle peut aussi se terminer quand
le flag \texttt{ZF} n'est pas d�fini. Les mn�moniques \texttt{loopnew} et
\texttt{loopnzw} forcent l'utilisation du registre \texttt{cx} alors que
\texttt{loopned} et \texttt{loopnzd} forcent l'utilisation du registre
\texttt{ecx}. Chaque instruction \texttt{loop} requiert une valeur imm�diate
indiquant l'adresse de destination qui ne peut �tre qu'un saut court
(saut maximum de 128 octets avant et 127 octets apr�s l'adresse suivant
l'instruction \texttt{loop}).

\texttt{jcxz}\emph{(destination)} saute au label indiqu� dans la destination si
la valeur du registre \texttt{cx} est �gale � z�ro, \texttt{jecxz} fait la m�me
chose mais v�rifie la valeur dans le registre \texttt{ecx}. Ces op�randes
suivent les m�mes r�gles que l'instruction \texttt{loop}

\texttt{int}\emph{(valeur)} d�clenche une interruption donn�e pas le num�ro
indiqu� dans l'op�rande valeur, celui-ci doit �tre compris entre 0 et 255.
La routine d'interruption se termine par l'instruction \texttt{iret} et
redonne le contr�le � l'instruction suivant le \texttt{int}. Le code
\texttt{int3} appelle l'interruption 3. \texttt{into} appelle l'interruption 4
si le flag \texttt{OF} est d�fini.

\texttt{bound}\emph{(destination,source)} v�rifie que la valeur destination
se trouve dans les limites d�finies par la source. Si la valeur contenue dans
le registre est plus petit que la borne inf�rieure ou plus grand que la borne
sup�rieure, on d�clenche l'interruption 5. L'op�rande destination est le
registre � tester, la source est une adresse m�moire des deux valeurs limites
sign�es. La taille des valeurs peut �tre de type \texttt{word} ou \texttt{dword}.
\begin{verbatim}    bound ax,[bx]       ; V�rifie les bornes pour un Mot
    bound eax,[esi]     ; V�rifie les bornes pour un Double Mot
\end{verbatim}

\begin{figure}
  \begin{center}
      \begin{tabular}{|c|c|c|}
      \hline
      Mn�monique&Conditions test�es&Description\\
      \hline\hline
      \verb"o"&OF = 1&d�bordement\\
      \hline
      \verb"no"&OF = 0&aucun d�bordement\\
      \hline
      \verb"c"&&retenue\\
      \verb"b"&CF = 1&inf�rieur\\
      \verb"nae"&&ni sup�rieur ni �gal\\
      \hline
      \verb"nc"&&aucune retenue\\
      \verb"ae"&CF = 0&sup�rieur ou �gal\\
      \verb"nb"&&pas inf�rieur\\
      \hline
      \verb"e"&ZF = 1&�gal\\
      \verb"z"&&z�ro\\
      \hline
      \verb"ne"&ZF = 0&pas �gal\\
      \verb"nz"&&diff�rent de z�ro\\
      \hline
      \verb"be"&CF or ZF = 1&inf�rieur ou �gal\\
      \verb"na"&&pas sup�rieur\\
      \hline
      \verb"a"&CF or ZF = 0&sup�rieur\\
      \verb"nbe"&&diff�rent de z�ro\\
      \hline
      \verb"s"&SF = 1&sign�\\
      \hline
      \verb"ns"&SF = 0&non sign�\\
      \hline
      \verb"p"&PF = 1&parit�\\
      \verb"pe"&&parit� paire\\
      \hline
      \verb"np"&PF = 0&aucune parit�\\
      \verb"po"&&parit� impaire\\
      \hline
      \verb"l"&SF xor OF = 1&inf�rieur\\
      \verb"nge"&&ni sup�rieur ni �gal\\
      \hline
      \verb"ge"&SF xor OF = 0&sup�rieur ou �gal\\
      \verb"nl"&&pas inf�rieur\\
      \hline
      \verb"le"&(SF xor OF) or ZF = 1&inf�rieur ou �gal\\
      \verb"ng"&&pas sup�rieur\\
      \hline
      \verb"g"&(SF xor OF) or ZF = 0&sup�rieur\\
      \verb"nle"&&ni inf�rieur ni �gal\\
      \hline
      \end{tabular}
      \caption{Conditions}
  \end{center}
\end{figure}
\newpage

\subsection{Les instructions d'E/S (Entr�es/Sorties)}
\texttt{in}\emph{(destination,source)} transfert un octet, mot ou double mot
d'un port d'entr�e vers le registre \texttt{al}, \texttt{ax} ou \texttt{eax}.
Les ports d'entr�e/sortie peuvent �tre adress�es directement, avec une valeur
imm�diate cod�e en octet dans l'instruction, soit une valeur imm�diate comprise
entre 0 et 255 ou indirectement via le registre \texttt{dx}.
La destination doit �tre un des registres suivant : \texttt{al}, \texttt{ax}
ou \texttt{eax}.
\begin{verbatim}    in al,20h           ; R�cup�re un octet du port 20h
    in ax,dx            ; R�cup�re un mot du port adress� par dx
\end{verbatim}

\texttt{out}\emph{(destination,source)} �crit un octet, mot ou double-mot dans
le port de destination, soit \texttt{al}, \texttt{ax} ou \texttt{eax}. On peut
indiquer le num�ro du port en utilisant les m�mes m�thodes que l'instruction
\texttt{in}.
\begin{verbatim}    out 20h,ax          ; Ecrit un mot dans le port 20h
    out dx,al           ; Ecrit un octet dans le port adress� par dx
\end{verbatim}

\subsection{Op�rations sur les cha�nes}
Les op�rations sur cha�ne de caract�res ne se font que sur un seul �l�ment
de la cha�ne. Un �l�ment de cha�ne peut �tre un octet, un mot ou un double-mot.
Chaque �l�ment de la cha�ne est adress� par les registres \texttt{si} et
\texttt{di} (ou \texttt{esi} et \texttt{edi}). Apr�s chaque op�ration sur
la cha�ne, \texttt{si} et/ou \texttt{di} (ou \texttt{esi} et/ou \texttt{edi})
sont automatiquement mis � jour en pointant sur le prochain �l�ment de la cha�ne.
Si \texttt{DF} (flag de direction) est �gal � z�ro, les registres d'index
sont incr�ment�s, si \texttt{DF} est �gal � un, ils sont d�cr�ment�s.
La valeur de l'incr�mentation ou de la d�cr�mentation d�pend de la taille
de l'�l�ment de la cha�ne, soit 1, 2 ou 4 (octet, mot ou double-mot).
Chaque instruction d'op�ration de cha�ne a un format court qui n'a aucune
op�rande et utilise \texttt{si} et/ou \texttt{di} quand le code est en 16 bits
ou \texttt{esi} et/ou \texttt{edi} quand le code est en 32 bits. \texttt{si}
et \texttt{esi} adressent par d�faut les donn�es dans le segment s�lectionn�
par \texttt{ds}, \texttt{di} et \texttt{edi} adressent toujours les donn�es
dans le segment s�lectionn� par \texttt{es}. La forme courte est obtenue en
ajoutant au mn�monique une lettre indiquant la taille de l'�l�ment de la
cha�ne, \texttt{b} pour un �l�ment de type octet, \texttt{w} pour un �l�ment
de type mot, et \texttt{d} pour un �l�ment de type double-mot. Le format
complet veut que les op�randes indique la taille de l'op�rateur ainsi que les
adresses m�moire, cela peut �tre \texttt{si} ou \texttt{esi} avec n'importe
quel prefix de segment, \texttt{di} ou \texttt{edi} avec toujours comme prefix de
segment \texttt{es}.
\texttt{movs}\emph{(destination,source)} transf�re le contenu de la m�moire
adress�e par exemple par \texttt{[ds:si]} (ou \texttt{[ds:esi]}), ou tout autre
segment, dans la m�moire adress�e par \texttt{[es:di]} (ou \texttt{[es:edi]})
avec toujours le segment \texttt{es}. La taille des op�randes peut �tre de
type octet, mot ou double mot. La destination doit �tre une m�moire adress�e
par \texttt{di} ou \texttt{edi} et la source est une m�moire adress�e par
\texttt{si} ou \texttt{esi}.
\begin{verbatim}    movs byte [di],[si]         ; Transfert d'octet de [si] vers [di]
    movs word [es:di],[ss:si]   ; Transfert d'un mot
    movsd                       ; Transfert d'un double-mot
\end{verbatim}

\texttt{cmps}\emph{(destination,source)} soustrait l'�l�ment de cha�ne
destination de l'�l�ment de cha�ne source et met � jour les flags \texttt{AF},
\texttt{SF}, \texttt{PF}, \texttt{CF} et \texttt{OF}, mais ne modifie aucun
des �l�ments compar�s. Si les deux �l�ments sont �gaux, \texttt{ZF} est d�fini
(= 1), autrement il est effac� (mis � z�ro). Il compare donc l'octet, le mot
ou le double-mot \texttt{[ds:si]} (ou \texttt{[ds:esi]}), ou tout autre
segment, � la m�moire adress�e par \texttt{[es:di]} (ou \texttt{[es:edi]})
avec toujours le segment \texttt{es}.
\begin{verbatim}    cmpsb                       ; Compare des octets
    cmps word [ds:si],[es:di]   ; Compare des mots
    cmps dword [fs:esi],[edi]   ; Compare des doubles mots
\end{verbatim}

\texttt{scas}\emph{(destination)} soustrait l'�l�ment de cha�ne destination du
registre \texttt{al}, \texttt{ax} ou \texttt{eax} (en fonction de la taille de
l'�l�ment de la cha�ne de caract�res) et met � jour les flags \texttt{AF},
\texttt{SF}, \texttt{ZF}, \texttt{PF}, \texttt{CF} et \texttt{OF}. Si les
valeurs sont identiques, \texttt{ZF} est d�fini, autrement il est effac�.
La destination doit �tre une m�moire adress�e par \texttt{di} ou \texttt{edi}.
\begin{verbatim}    scas byte [es:di]           ; Scanne un octet
    scasw                       ; Scanne un mot
    scas dword [es:edi]         ; Scanne un double mot
\end{verbatim}

\texttt{lods}\emph{(source)} met dans \texttt{al}, \texttt{ax} ou \texttt{eax}
le contenu de la m�moire indiqu� dans la source. La source doit �tre une m�moire
adress�e par \texttt{si} ou \texttt{esi} avec n'importe quel segment.
\begin{verbatim}    lods byte [ds:si]           ; Charge un octet
    lods word [cs:si]           ; Charge un mot
    lodsd                       ; Charge un double mot
\end{verbatim}

\texttt{stos}\emph{(destination)} met la valeur de \texttt{al}, \texttt{ax} ou
\texttt{eax} dans la m�moire indiqu�e dans la destination. Cette instruction
suit les m�me r�gles que l'instruction \texttt{scas}.

\texttt{ins}\emph{(destination,source)} transf�re un octet, mot ou double mot
d'un port d'entr�e adress� par le registre \texttt{dx} � l'�l�ment cha�ne dans
la destination. La destination doit �tre une m�moire adress�e par \texttt{di} ou
\texttt{edi}, la source doit �tre le registre \texttt{dx}.
\begin{verbatim}    insb                        ; Lit un octet
    ins word [es:di],dx         ; Lit un mot
    ins dword [edi],dx          ; Lit un double mot
\end{verbatim}

\texttt{outs}\emph{(destination,source)} transf�re un �l�ment cha�ne de
caract�re de la source vers le port de sortie adress� par le registre
\texttt{dx}. La destination doit �tre le registre \texttt{dx} et la source
doit �tre une m�moire adress�e par \texttt{si} ou \texttt{esi} avec n'importe
quel segment.
\begin{verbatim}    outs dx,byte [si]           ; Ecrit un octet
    outsw                       ; �crit un mot
    outs dx,dword [gs:esi]      ; �crit un double mot
\end{verbatim}

Les pr�fixes de r�p�tition \texttt{rep}, \texttt{repe}/\texttt{repz} et
\texttt{repne}/ \texttt{repnz} permettent des op�rations de r�p�titions sur des
cha�nes. Quand une instruction a le pr�fixe de r�p�tition, celle-ci est r�p�t�e
 chaque fois en utilisant un �l�ment diff�rent de la cha�ne de caract�res.
 Cette r�p�tition se termine quand une des conditions sp�cifiques au pr�fixe
 est satisfaisante. Apr�s chaque op�ration, ces trois pr�fixes d�cr�mentent
 automatiquement le registre \texttt{cx} ou \texttt{ecx} (en fonction de
 l'adressage en 16 ou 32 bits) et se r�p�te jusqu'� ce que \texttt{cx} ou
 \texttt{ecx} soit �gal � z�ro.  \texttt{repe}/\texttt{repz} et \texttt{repne}/
\texttt{repnz} sont exclusivement utilis�s avec les instructions \texttt{scas}
et \texttt{cmps} (d�crites plus haut). Quand ces pr�fixes sont utilis�s, la
r�p�tition de l'instruction suivante d�pend aussi du flag z�ro (\texttt{ZF}),
\texttt{repe} et \texttt{repz} arr�te l'execution du code quand \texttt{ZF}
est �gal � z�ro, \texttt{repne} et \texttt{repnz} stoppe l'ex�cution quand
\texttt{ZF} est d�fini (mis � un).
\begin{verbatim}    rep  movsb        ; Transfert de plusieurs double mots
    repe cmpsb        ; Compare des octets tant qu'ils sont �gaux
\end{verbatim}

\subsection{Instructions de contr�le du Flag}
Les instructions de contr�le du flag fournissent une m�thode pour changer
directement l'�tat d'un bit dans le registre flag. Toutes les instructions
d�crites dans cette section n'ont pas d'op�randes.

\subsection{Op�rations conditionnelles}

\subsection{Instructions diverses}

\subsection{Syst�me}

\subsection{FPU}

\subsection{MMX}

\subsection{SSE}

\subsection{SSE2}

\subsection{SSE3}

\subsection{AMD 3DNow!}

\subsection{Le mode x86-64}

\section{Directives de contr�le}

\subsection{Constantes num�riques}

\subsection{Assemblage conditionnel}

\subsection{Blocs d'instructions r�p�t�s}

\subsection{Adressage des espaces}

\subsection{Autres directives}

\subsection{Passages multiples}

\section{Directives du pr�-processeur}

\subsection{Inclusion de fichiers}

\subsection{Constantes symboliques}

\subsection{Macro-Instructions}

\subsection{Structures}

\subsection{Macro-Instructions r�p�t�es}

\subsection{Phase du pr�-processeur conditionnelle}

\subsection{Ordre du process}

\section{Directives du format}

\subsection{Executable MZ}

\subsection{Portable Executable : PE}

\subsection{Common Object File Format : COFF}

\subsection{Executable and Linkable Format : ELF}
      % Chapitre Jeu d'Instructions
\chapter{Programmation Windows}

\section{Ent�tes de base}

\subsection{Structures}

\subsection{Imports}

\subsection{Proc�dures}

\subsection{Exports}

\subsection{Component Object Model : COM}

\subsection{Ressources}

\subsection{Encodage du texte}

\section{Ent�tes �tendues}

\subsection{Param�tres de proc�dure}

\subsection{Structurer la source}
     % Chapitre Programmation Windows

%\backmatter                   % �pilogue

%\addcontentsline{toc}{chapter}{Index}
%\printindex                   % Cr�� l'index
%\printglossary

\end{document}
